\label{chap:kapitel3_2}
\section{Verbesserter Algorithmus von Wetherell und Shannon}
Wetherell und Shannon stellen in ihrem Paper einen weiteren, verbesserten Algorithmus zum Zeichnen von Bäumen vor, welcher jedoch
ausschließlich Binärbäume zeichnen kann. Dieser Algorithmus weist die Nachteile des naiven Algorithmus nicht mehr auf.
Dafür definieren sie zwei weitere Anforderungen, die der Algorithmus erfüllen soll.   

\begin{quotation}
	\textit{Aesthetic 2:} In a binary tree, each left son should be positioned
	left of its father and each right son right of its father.\cite[]{q1}
\end{quotation}

In einem Binärbaum hat jeder Knoten maximal ein linkes und maximal ein rechtes Kind. Daher ist es auch logisch, dass jedes linke Kind 
links vom Vater und jedes rechte Kind rechts vom Vater positioniert werden soll. Zudem soll jeder Vater zentriert über seinen Kindern
stehen. Dieses Verhalten legen Wetherell und Shannon in einer weiteren Anforderung fest.

\begin{quotation}
	\textit{Aesthetic 3:} A parent should be centered over its children.\cite[]{q1}
\end{quotation}


\subsection{Vor- und Nachteile}
Durch das Erfüllen der Anforderung, dass jeder Vater über seinen Kindern zentriert werden soll, kann der verbesserte Algorithmus gegen
die Anforderung an das physikalische Limit verstoßen. Dies geschieht, da der Algorithmus die Zentrierung der Väter erzwingt.
Abbildung \todo{Abbildung einfügen} zeigt jenes Verhalten. Daraus schließen die beiden Autoren auf folgendes Theorem:

\begin{quotation}
	\textit{Theorem (Uglification):} Minimum width drawings exist which violate Aesthetic 3 by arbitrary amounts.\cite[]{q1}
\end{quotation}

Bei der schmaleren Variante des Baumes auf Abbildung XXX ist der Vater von Knoten A nicht zentriert über beiden Kindern und verstößt somit 
gegen Aesthetic 3. Die weitere Variante verstößt gegen das physikalische Limit, da der Baum nicht maximal schmal ist. 
Dies stellt hier einen Trade-off zwischen den beiden Anforderungen dar. Es ist somit nicht möglich einen maximal schmalen Baum zu zeichnen,
der auch Aesthetic 3 erfüllt. 

Zudem ist der verbesserte Algorithmus nicht in der Lage beliebige Bäume zu zeichnen, sondern ausschließlich Binärbäume. Jedoch kann er durch
einige geeignete Erweiterungen so modifiziert werden, dass er auch in der Lage wäre beliebige Bäume zu zeichnen.
