\chapter{Algorithmen zum Zeichnen von Bäumen}
\label{chap:kapitel3}

\section{Naiver Algorithmus von Wetherell und Shannon}

Das Paper “Tidy Drawings of Trees” von Charles Wetherell und Alfred Shannon aus dem Jahre 1979, 
welches im IEEE Trans. Softw. Eng. erschienen ist, handelt von verschiedenen Algorithmen zum Zeichnen von Bäumen.
Der erste Algorithmus, der von den beiden Autoren beschrieben und vorgestellt wird, ist ein naiver Algorithmus 
zum Zeichnen von Bäumen. Dieser Algorithmus soll dabei zwei Anforderungen erfüllen. Die erste Anforderung wird dabei
an die Ästhetik des gezeichneten Baumes gestellt. Alle Knoten, die dieselbe Höhe haben, sollen sich auf einer horizontalen
Linie befinden. Jede Höhe hat dabei eine Linie, auf welcher sich die Knoten befinden sollen und diese Linien sollen alle
parallel zueinander sein. Außerdem soll der Algorithmus beim Zeichnen eines Baumes ein physikalisches Limit einhalten.
Das bedeutet, dass der Algorithmus möglichst schmale Bäume zeichnen soll. Jedoch wird die Höhe des Baumes durch diese Anforderungen
nicht eingeschränkt. Stattdessen bestimmt der Baum selbst seine Höhe. \cite[]{q1}

\subsection{Ablauf}

\subsection{Vor- und Nachteile}

\section{Verbesserter Algorithmus von Wetherell und Shannon}

\subsection{Ablauf}

\subsection{Vor- und Nachteile}

\section{Algorithmus von Reingold und Tilford}

\subsection{Ablauf}

\subsection{Implementierung in Java}

\subsection{Vor- und Nachteile}

\subsection{Modifizierung des Algorithmus}
