\chapter{Konklusion}
\label{chap:kapitel5}
Mit dieser Arbeit sollte ein Verständnis für verschiedene Algorithmen zum Zeichnen von Bäumen geschaffen werden. Dabei wurde
begonnen bei einem naiven Algorithmus und hingearbeitet zu jeweils zwei verbesserten Algorithmen. Hierfür wurde besonders
auf den Ablauf der einzelnen Algorithmen eingegangen. Zudem wurde für jeden Algorithmus eine mögliche Implementierung in Java
dargestellt sowie beispielhaft einige Bäume mit diesen Implementierungen gezeichnet. Außerdem wurde bei jedem Algorithmus
auf die spezifischen Vor- und Nachteile des jeweiligen Algorithmus eingegangen sowie ein abschließender Vergleich gezogen.\\

Nach der Betrachtung des naiven Algorithmus wird deutlich, dass dieser keine ästhetisch ansprechenden Bäume zeichnet. Weiterführend wurde ein 
verbesserter Algorithmus von \ac{WS} vorgestellt. Die durch den Algorithmus gezeichneten Bäume halten sich an das physikalische Limit 
sowie an die ersten beiden Anforderungen an die Ästhetik. Jedoch priorisiert dieser Algorithmus die Anforderung an das physikalische
Limit, was zur Folge hat, dass ästhetisch unansprechende Bäume gezeichnet werden. Ferner besitzt dieser Algortihmus nicht die Möglichkeit
Spiegelbilder von Bäumen korrekt zu zeichnen. Der Algorithmus von \ac{TR} greift dies auf und stellt eine weitere Möglichkeit zum Zeichnen
von Bäumen dar. Dieser erfüllt alle Anforderungen an die Ästhetik von gezeichneten Bäumen, jedoch teilweise auf Kosten des 
physikalischen Limits. 


