\section{Data Preparation}

Mit den Erkenntnissen aus der vorherigen Phase beginnt nun die Aufbereitung der Daten. Jede Komponente bzw. jedes System hat im Project RA Application Conditions einen Ordner \glqq 10\char95 Project
Answers\grqq{}. Nutzt ein Projekt nun beispielsweise das Achszähler-System AzS350U und hat die Anwendungsregeln des Systems bewertet, dann müssen diese bewerteten Anwendungsregeln in ein neues Modul 
in dem \glqq 10\char95 Project Answers\grqq{} Ordner des Achszähler-Systems geschrieben werden. Mit den Modulen, die die bewerteten Anwendungsregeln der einzelnen Komponenten und Systeme beinhalten,
kann dann das KI-System angelernt werden. 

Um das zu erreichen, müssen zunächst die 115 Module genauer untersucht werden. Dabei muss geprüft werden, ob diese Module die Anwendungsregeln so bewertet haben, wie es das Process Manual vorgibt.
Dafür muss die Iteration aus \ref*{lst:searchInLinks} erweitert werden. Es muss zusätzlich geprüft werden, ob das Modul die Attribute REQ Status bzw. REQ Progress besitzt und ob diese auch 
ausgefüllt sind. 

\begin{lstlisting}[caption={Suche nach bewerteten Anwendungsregeln},captionpos=b, label = lst:searchARSolutions]                                                       
if(hasAttribute("REQ Status", mod) 
&& hasAttribute("REQ Statement", mod)){  
    szProgress = o."REQ Status""";
    szStatement = o."REQ Statement";
    if(length(szProgress)>1 && length(szStatement)>1){ 
        put(slResults, fullName mCheck, fullName mCheck)
    }
}else if (hasAttribute("REQ Progress", mod) 
&& hasAttribute("REQ Statement", mod)){
    szProgress = o."REQ Progress""";
    szStatement = o."REQ Statement";
    if(length(szProgress)>1 && length(szStatement)>1){ 
        put(slResults, fullName mCheck, fullName mCheck)
        // ...
\end{lstlisting}

Übrig bleiben danach 41 Module aus 34 Projekten, welche nicht nur die Anwendungsregeln importiert, sondern auch nach den Vorgaben des Process Manuals bewertet haben.  

% TODO

- Zwischenschritt: für jedes Projekt ein Modul in Sandbox
- dabei darauf achten, dass nur relevante Attribute kopiert werden und keine old versions
- diese dann aufteilen, auf einzelne Komponenten/Systeme durch Links
- die neuen Module dann in Projects Answers einfügen
- FERTIG