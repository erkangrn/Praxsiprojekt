%\chapter*{Überblick}
\section*{Kurzfassung}
Bäume sind eine besondere Form von Graphen und stellen eine Datenstruktur dar. Diese bestehen aus zwei Elementen, 
den Knoten und Kanten. In dieser Arbeit wird sich mit drei verschiedenen Algorithmen zum Zeichnen von Bäumen
beschäftigt und vorgestellt. Diese Algorithmen ermöglichen es, 
Bäume im Ebenen-Layout zu zeichnen. Zwei dieser Algorithmen kommen von Wetherell und Shannon, der Dritte von Reingold und Tilford.
Im Folgenden werden die Abläufe dieser Algorithmen beschrieben und im Späteren in Java implementiert. Zuletzt werden diese Algorithmen 
an einem konkreten Beispiel verglichen und daran deren zuvor beschriebenen Vor- und Nachteile deutlich gemacht.

\vfill\vfill\vfill\vfill\vfill\vfill
\section*{Abstract}
Trees are a special form of graphes. They represent a data structure and consist of two elements,
the nodes and edges. Three different algorithms are discussed and presented here. These algorithms make it possible to
draw trees in a planar layout. Two of these algorithms come from Wetherell and Shannon, the third from Reingold and Tilford.
In the following, the processes of these algorithms are described and later implemented in Java. Furthermore, these algorithms are
compared using a specific example. Finally the advantages and disadvantages, which will be described beforehand, will be shown on that 
particular example.
\vfill\vfill\vfill\vfill\vfill\vfill